\usepackage{fancybox}
\usepackage[progressbar=frametitle,background=light,block=fill]{theme/beamerthememetropolis2}
%\usepackage[progressbar=frametitle,background=dark,block=fill]{theme/beamerthememetropolis2}
\usepackage{amssymb}
\usepackage{amsmath}
\usepackage{booktabs}
\usepackage{esint}
\usepackage{multicol}
\usepackage{graphicx}
\graphicspath{{img/}}
\usepackage{tcolorbox}
\usefonttheme{professionalfonts}
\usepackage{physics}
\hypersetup{unicode}
%\usepackage[boldfont]{xeCJK}
%\setCJKmainfont{AozoraMinchoRegular.ttf}
\usepackage{empheq}
\usepackage[T1]{fontenc}
\usepackage{mathpazo}
\usepackage{kerkis}
\usepackage{esint}

\makeatletter
\newcommand\Cshadowbox{\VerbBox\@Cshadowbox}
\def\@Cshadowbox#1{%
  \setbox\@fancybox\hbox{\fbox{#1}}%
  \leavevmode\vbox{%
    \offinterlineskip
    \dimen@=\shadowsize
    \advance\dimen@ .5\fboxrule
    \hbox{\copy\@fancybox\kern.5\fboxrule\lower\shadowsize\hbox{%
      \color{mLightBrown}\vrule \@height\ht\@fancybox \@depth\dp\@fancybox \@width\dimen@}}%
    \vskip\dimexpr-\dimen@+0.5\fboxrule\relax
    \moveright\shadowsize\vbox{%
      \color{mLightBrown}\hrule \@width\wd\@fancybox \@height\dimen@}}}
\makeatother

\setbeamertemplate{blocks}[rounded][shadow]

\newcommand{\maketitlelight}{\maketitle%

\setbeamertemplate{background canvas}{%
    \includegraphics[width=\paperwidth, height=\paperheight]{theme/back.png}
}}

\newcommand{\maketitledark}{\maketitle%

\setbeamertemplate{background}{%
    \includegraphics[width=\paperwidth, height=\paperheight]{theme/back2.png}
}
}

\usepackage{theme/pdfpc-commands}

\newcommand\Wider[2][3em]{%
\makebox[\linewidth][c]{%
  \begin{minipage}{\dimexpr\textwidth+#1\relax}
  \raggedright#2
  \end{minipage}%
  }%
}

\setbeamercovered{transparent}

\usepackage{ruby}
\renewcommand{\rubysep}{-0.5ex}